\documentclass[a4paper, 12pt]{article}
\usepackage{comment} 
\usepackage{fullpage}
\usepackage[hidelinks]{hyperref}
\usepackage{amsmath}
\usepackage{environ}
\usepackage{tabto,enumitem}

\begin{document}
\noindent
\large\textbf{SOEN 6011} \hfill \textbf{Aniket Tailor}\\
\large\textbf{Software Engineering Processes} \hfill \textbf{40195068} \\
Function 8 : Standard Deviation $\sigma$ \hfill Date: 5 August 2022\\
\normalsize Problem 2 \\
\section*{Assumptions}
Any variable, as long as it can be sorted, can have its variance and standard deviation determined. However, the standard deviation is only a useful indicator of dispersion for a measurement variable when the data have a symmetrical distribution, which is frequently a normal distribution. If these presumptions are not true, using the standard deviation to show the variability of observations in range plots and box-and-whisker plots is misleading. This assumption is also a prerequisite for assumptions on the percentage of observations that fall within the range of agreement.

\section*{Requirements}
\begin{enumerate}[noitemsep]
        \item \textbf{First Requirement}
        \begin{itemize}[noitemsep]
            \item \textbf{ID = } F1
            \item\textbf{Type = } Functional Requirement
            \item\textbf{Version = } 1.0
            \item\textbf{Priority = } High
            \item\textbf{Description = } Standard deviation only deals with numbers and not strings.
        \end{itemize}
        \item \textbf{Second Requirement}
        \begin{itemize}[noitemsep]
            \item \textbf{ID = } F2
            \item\textbf{Type = } Functional Requirement
            \item\textbf{Version = } 1.0
            \item\textbf{Priority = } High
            \item\textbf{Description = } Standard deviation is the square root of Variance, hence it's value should not be negative.
        \end{itemize}
        \item \textbf{Third Requirement}
        \begin{itemize}[noitemsep]
            \item \textbf{ID = } F3
            \item\textbf{Type = } Functional Requirement
            \item\textbf{Version = } 1.0
            \item\textbf{Priority = } Moderate
            \item\textbf{Description = }  It's important to keep in mind that the standard deviation occurs together with the variance.
        \end{itemize}
        
        \newpage
        \item \textbf{Fourth Requirement}
        \begin{itemize}[noitemsep]
            \item \textbf{ID = } F4
            \item\textbf{Type = } Functional Requirement
            \item\textbf{Version = } 1.0
            \item\textbf{Priority = } High
            \item\textbf{Description = }   For calculating the mean, natural and real numbers should be considered.
        \end{itemize}
        \item \textbf{Fifth Requirement}
        \begin{itemize}[noitemsep]
            \item \textbf{ID = } F5
            \item\textbf{Type = } Functional Requirement
            \item\textbf{Version = } 1.0
            \item\textbf{Priority = } High
            \item\textbf{Description = }   For calculating standard deviation, at-least two numbers should be given as input.
        \end{itemize}
        \item \textbf{Sixth Requirement}
        \begin{itemize}[noitemsep]
            \item \textbf{ID = } F6
            \item\textbf{Type = } Functional Requirement
            \item\textbf{Version = } 1.0
            \item\textbf{Priority = } High
            \item\textbf{Description = }   If all the data values are same then the standard deviation should be 0.
        \end{itemize}
        \item \textbf{Seventh Requirement}
        \begin{itemize}[noitemsep]
            \item \textbf{ID = } F7
            \item\textbf{Type = } Non Functional Requirement
            \item\textbf{Version = } 1.0
            \item\textbf{Priority = } Low
            \item\textbf{Description = }   Code should be well indented and easy to understand with proper documentation. 
        \end{itemize}
        \newpage
    \end{enumerate}


\end{document}
